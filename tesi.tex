\documentclass[]{article}

%opening
\title{VPN Decentralizzate}
\author{Jacopo Federici}

\begin{document}
	
	\maketitle
	
	\begin{abstract}
		
	\end{abstract}
	
	\section{Index}
	
	\begin{enumerate}
			\item Introduction
				\begin{enumerate}
					\item the world, now (stato attuale, attacchi etc)
					\item the needing of having privacy, what the world can offer (soluzioni e stato dell'arte)
					\item just more, we are our trust (concetti di consenso, fiducia)
				\end{enumerate}
			\item Analyze malicious contexts.
				\begin{enumerate}
					\item Basic sniffic (Wireshark, tcpdump)
					\item Craftberry, a demo tool
					\begin{itemize}
						\item liv 2: arp, vlan tagging
						\item liv 3: ip, icmp
						\item liv 4: tcp/udp
						\begin{itemize}
							\item copia x volte dei pacchetti (TCP/UDP)
							\item gonfiamento dei pacchetti nei campi che poi verranno corretti dai check (TCP/UDP)
							\item modifica del payload se in chiaro (TCP/UDP)
							\item azione positiva sul payload (UDP), ad esempio cifratura
						\end{itemize}
						\item liv 5+: http/imap/dns/ntp/etc
						\begin{itemize}
							\item analisi e modifica dei contenuti a livello del singolo protocollo (ad esempio sostituzione di tutte le immagini con una immagine di default)
						\end{itemize}
					\end{itemize}	
				\end{enumerate}
			\item Design solutions for malicious context.
				\begin{itemize}
					\item invia la richiesta a due nodi e le confronta (essenziale possibilità di connessione a due nodi in contemporanea: due o più container docker mappati su porte diverse con uno script che invia la richiesta a più container e confronta le risposte)
					\item come precedente ma con la gestione degli hash invece che di tutto il dato
				\end{itemize}
			\item Design integration with reputation systems.
			
	\end{enumerate}
	
	\section{Introduction}
	A virtual private network (VPN) allows to extend a \textit{private} network across a \textit{public} network. It is created by establishing a virtual point-to-point connection through the use of dedicated circuits or with tunneling protocols over existing networks. The data passed inside the tunnel is usually encrypted and this quality has been on the ground for its widely diffusion.\\
	
	Currently, there are many commercial solutions that allow users to have a private connection between their computer and the company servers across the world. In this way the user can have access to the internet resources from the country he prefers, skipping firewalls and other censure methods applied by some countries, such as China with the Great Fire Wall.\\
	There are many security concerns about those commercial solutions, such as the ability of the company to keep logs, inspect the traffic or sell statistic data obtained by the users.\\
	Starting by these problems, some open source projects have, lately, been born. Their goal is to revolution the idea of VPN and its uses, building a rich network of peers where two nodes can connect to each others through a VPN. The Network provides technical solutions to avoid data sniffing or crafting, ensure identity anonymisation and secure payments.\\
	
	Before to deep into the technical aspects why such a projects haven't been developed before, it's important to understand the scenario and the needs of the actors. We use an example to explain it clearly.\\
	Alice wants to freely surf on social networks but her country blocks the major of them. Technically speaking, what she needs is to exit into internet with an IP address from some other country where those service are not blocked, exactly what a VPN service provides, and she welcome to pay for it. 
	The exit point, in our example, is Bob who agrees to let Alice connect to him to exit into internet, and he wants to be paid for it.\\
	The Networks presented before create the ideal network platform where a consumer, Alice, and a producer, Bob, can meet each other, deal with a contract and rate the opposite quality at the contract closing.\\
	Generalizing, we can assume that bigger is the network and better are all of those aspects we are striving to get, such as anonymisation, anti-sniffing and anti-crafting solutions and so on.\\
	In this context another actor stands watching everything happening. He has a black hat, glasses and a cup of coffee close to him. His name is Mallory and, as we'll describe later, he's trying to find ways to use the system in a manner the system has not been designed. In other words, he's hacking the system. At least, tryin' to.\\
	
	
	It is natural that concepts as "consensus" and as "distributed" gain value and it's undoubted the main technology to manage such kind of situation is the blockchain in the Ethereum variant thanks to its contract mechanism.\\
	
	
	\section{Projects}
	At the time we are coding, the main projects are Mysterium Network, Privatix, Substratum and Sentinel. We explored all of them and we discovered the four different approaches they have to reach the goal of building a distributed network based on blockchain with a discovering system, a crypto coin payment method, a mobile and desktop application. We can safely say the most advanced project is Mysterium Network, in terms of development and design. Furthermore it's 100\% open source, that for this kind of project is an important quality which conditions the users adoption and market diffusion.

	\subsection{concetti introduttivi}
	 - blockchain in linea di massima\\
	 - ethereum in linea di massima e descrizione degli smart contracts\\
	 - proof of work, proof of stake\\
	 - reputazione in una net (Eigentrust)\\
	 - - cercare applicazione pratica di Eigentrust\\
	 - descrizione di BFT\\
	
	Therefore, we decide to use Mysterium Project to implement our proof of concepts.\\
	All of the projects have in common the idea behind the business model based on the users' abilities to sell their internet or to use a VPN service, depending on their needs and, accordingly, on their role into the network.
	
	
	\subsection{Mysterium Network}
	Mysterium Network è un progetto open source per la realizzazione di un network P2P decentralizzato che usa multi-hops, relays e micropagamenti integrati basati su token.\\
	
	Nel contesto che si vuole descrivere si identificano due entità: un Client A che usa il servizio ed un Agent B che lo fornisce.\\
	Mysterium utilizza un Service Provider, basato su blackchain ethereum, che registra le intenzioni del Client A di pagare l’Agent B per il servizio che B offre ad A.\\
	
	Una promessa è una quadrupla contenente i seguenti elementi:
	
	\begin{itemize}
	\item P è firmato da A
	\item Una promessa P può essere aggiornata in caso di prosecuzione del servizio
	\item B può bloccare l’aggiornamento della promessa P
	\item P è valida se è firmata da A e se non è stata aggiornata dopo essere stata bloccata
	\item L’attributo SN incrementa ad ogni modifica
	\item Per identificare uno stato non valido basta verificare il SN se è maggiore del SN nel momento del blocco
	\item La chiusura di un contratto P, con P valido, termina con il pagamento della somma MYST da A a B
	\item Ogni transazione ethereum per il pagamento di MYST, generata a seguito della chiusura di un contratto, comporta il pagamento di fees e gas
	\item Si crea un equilibrio tra trasferimento di poco denaro ma frequente e trasferimento di molto denaro ma poco frequente
	\begin{itemize}
	\item Il primo comporta un totale di fees maggiore rispetto al secondo ma corrisponde ad una più alta garanzia di pagamento rispetto al secondo (se un client A non paga, non paga poco denaro, invece che tanto) [merita di essere analizzato meglio]
	\end{itemize}
	\item Le promesse possono essere condivise con più Agents B senza perdita di valore
	\begin{itemize}
	\item La condivisione consente di evitare situazioni in cui un Client A con pochi denari instauri molteplici promesse (false) con diversi Agents B: [DA CAPIRE MEGLIO]
	\end{itemize}
	\item Gli Agents B possono decidere il grado di rischio a cui esporsi (ovvero al rischio che i Clients non paghino)
	\end{itemize}	

	La promessa non garantisce, di per sé, il pagamento: infatti se A non ha abbastanza MYST nell'account per il pagamento, la sua promessa non viene mantenuta e la sua identità, in successive promesse, viene compromessa.
	Si utilizza quindi un registro, chiamato Identity Registry (IR) che tiene traccia delle identità [capire meglio che dati dell’identità vengono usati] che usano il servizio, attraverso il quale scoraggiare l’emissione di promesse non mantenibili.
	
	
	
	\subsection{Privatix}	
	Privatix utilizza due tipi diversi di smart contract con due funzionalità diverse.	
	\begin{itemize}
	\item PTC: Privatix Token Contract è usato per scambiare token, upgrade o nuovi contratti di servizio
	\item PSC: Privatix Service Contract è usato per immagazzinare balance (depositare e ritirare) e per la gestione del channel...
	\end{itemize}	
	I coin PRIX possono essere acquistati e venduti sono con PTC. I servizi Privatix, invece, possono essere pagati solo usando PSC.
	L’idea quindi è quella di usare lo smart contract PSC per le operazioni interne e lo smart contract PTC per le operazioni con l’esterno, con il fine ultimo di aumentare la sicurezza.\\	
	Attacchi Sybil: Attualmente non esistono tecnologie per mitigare l’attacco che non prevedano l’introduzione di qualche svantaggio: così dicono loro.
	Separano l’attacco in due categorie, in base all’entità malevola.	\\
	Agents Malevoli: A tutti gli agenti è richiesto di registrare il servizio offerto nella blockchain Ethereum e di depositare una somma di denaro che è possibile ritirare in seguito così da diminuire la presenza di Smart Operation (SO) fasulle. Il deposito è proporzionale alla quantità di servizio offerto.
	Se l’Agent malevolo posiziona SO fasulle, gli sarà richiesto di posizionare la stessa quantità di denaro che il Client ha posizionato quando ha accettato l’offerta dell’Agent.
	Le SO hanno associata una età e quindi gli Agents devono mantenere la loro SO viva, altrimenti i Clients la considereranno irrilevante.
	Ad ogni ri-notifica degli Agents, essi devono bloccare (posizionare) nuovamente la stessa quantità di denaro.\\	
	Quando un Client crea uno state channel con un Agent e non riceve il servizio accordato, allora chiuderà lo state channel con lo stato Uncooperative. In questo caso l’Agent sarà notificato da un evento blockchain così da intaccare la sua reputazione e forzare l’Agent malevolo a creare una nuova identità.\\	
	Se l’Agent malevolo opera sia come Client che come Agent per incrementare il numero operazioni Cooperative, si troverà di fronte al pagamento di fees ed ogni volta che opera come Client diminuirà la quantità di servizio a disposizione come Agent.\\	
	Client Malevoli:	Un Client può creare uno state channel e non inviare la prova di possesso della quantità di denaro necessaria a pagare il servizio. Ma al Client è richiesto di depositare quella cifra che sarà bloccata per tutto il periodo del channel. Per ottenere il denaro indietro è necessario chiudere il channel, ma la transazione consuma Ethere e quindi costa.
	Quindi se un Client vuole danneggiare un Agent le sue possibilità sono ridotte e costose, inoltre sarà segnata Uncooperative, a suo sfavore, una chiusura di un channel.\\	
	Sia i Client che gli Agent ottengono la chiusura del channel in modo Cooperative o Uncooperative e la chiusura viene registrata in blockchain con associati gli indirizzi di entrambi.
	Il sistema blockchain consente di analizzare le chiusure Cooperative e Uncooperative di tutti così da verificare la reputazione a cascata.	\\
	Quindi: Fondamentalmente non hanno trovato un modo per eliminare al 100\% attacchi Sybil ma cercano, con i sistemi sopra descritti, di limitare al massimo le possibilità di azione e la loro efficacia.
	
	\subsection{Sentinel}
	
	\subsubsection{Architettura}
	\begin{itemize}
	\item Identity Chain: Anonymous User ID (AUID) creati e memorizzati in una blockchain ad hoc.
	\item Service Chain: Rete sicura su cui si appoggiamo le operazioni di gestione e sono usati i token \$SENT/\$SST per le transazioni (\$SENT sono usati esternamente e sono usati comprare \$SST che sono usati internamente, rapporto 1:1)
	\item Transaction Chain: gestisce i pagamenti ed i relativi processi di invio delle transazioni al Sentinel Transaction Pool.
	\end{itemize}	
	L’unico punto di accesso ai servizi di Sentinel è l’AUID.\\
	Il sistema è basato sulla reputazione tale per cui più è alta e maggiore è l’accesso ai servizi. C’è un meccanismo di guadagno di denaro in funzione della reputazione.\\
	Il consenso distribuito riconosce e mitiga velocemente cattivi attori.\\
	La soluzione adottata per mitigare il problema di attori cattivi sul nodo di uscita è la stessa adottata dalle reti come TOR, ovvero l’uso di un tecniche di routing e ritrasmissione dei pacchetti per assicurare che sia la sorgente che la destinazione non siano rilevati.\\
	Si sta attualmente sviluppando la rete di ritrasmissione nella quale i partecipanti posso decidere il ruolo (perché?).	
	\subsubsection{Idee di Sentinel}	
	Al contrario delle altre soluzioni loro vogliono creare una piattaforma su cui appoggiare una serie di prodotti e servizi oltre alla dVPN: dChat, dVoIP, dFiles, dCDN, dDNS, dCompute, chiamate dAPPs.
	Hanno intenzione di realizzare un internet box hardware da posizionare tra l’accesso ad internet (gateway) ed il router.\\	
	Attualmente Sentinel si appoggia a Cosmos, piattaforma che consente la creazione di una blockchain decentralizzata di criptovaluta. Cosmos è costruito su Tendermin che fornisce il livello networking e consensus di una blockchain (quindi Cosmos è l’application di cryptocurrencies)\\	
	Quindi, per quanto riguarda Sentinel, il problema del consenso BFT è risolto dalla piattaforma su cui si appoggia.
	
	\subsubsection{Tendermint}
	Tendermint è un software per la replicazione in modo sicuro e consistente di una applicazione su una moltitudine di macchine. Con in modo sicuro si intende che Tendermint funziona anche se 1/3 delle macchine fallisce arbitrariamente.
	https://tendermint.com/docs/introduction/what-is-tendermint.html
	
	\subsection{Substratum}
	\subsection{caratteristiche comuni}
	\subsection{BTF}
	\subsection{Attacchi di tipo...}
		
	\section{problemi individuati}
	\begin{enumerate}
		\item valutazione della reputazione dell'agent da parte del client
		\item valutazione della reputazione dell'agent da parte della Net
	\end{enumerate}
	
	\section{soluzioni proposta}
	\subsection{Proposte per la valutazione della reputazione dell'agent da parte del client}
	
	Le seguenti proposte sono metodi di verifica del corretto comportamento ed hanno come scopo finale la valutazione della reputazione degli agent
	\begin{itemize}
		\item Dati civetta: modifiche al client che con scadenze richiede una o più risorse dal valore noto e verifica che siano integre. Le scadenze possono essere regolari/random/all’inizio frequenti/dipendenti dalla reputazione dell’agent. È necessario avere delle risorse distribuite e disponibili: potrebbero essere i nodi stessi della rete (altri agent).
		\item Dati duplicati: modifiche al client che implementa la possibilità di connessione con più agent. Dopo la richiesta il client compara i dati ottenuti dalle due fonti
		\begin{itemize}
			\item Hash *: (soluzione aggiuntiva) I nodi sono generalmente in posizioni migliori dei client in termini di velocità. L’idea è quella di far generare un hash dei dati che il client richiede ad un altro nodo fuori dal servizio primario, così da comparare gli hash e non tutto il dato.
			Inoltre, se la richiesta la faccio a molti più nodi posso intrinsicamente verificare quali modificano i dati nel network.
		\end{itemize}
		\item Applicazione di modelli di reputazione (Eigentrust): attualmente non studiato
	\end{itemize}

	\pagebreak
	
		{\Large Valutazione di un nuovo utente della rete}
	\paragraph{Contesto}
	Rete VPN decentralizzata attiva e con connessione multi-hop tra una generica coppia Agent-Client.
	
	\paragraph{Problema 1: identificare gli Agent in posizione endnode che inviano contenuto non originale}
	In una rete VPN decentralizzata gli Agent endnode hanno il compito di recuperare una risorsa web e restituirla al Client attraverso la rete VPN instaurata. L'Agent, che quindi ha accesso intrinseco al dato ottenuto, cifrato o meno, può modificarlo prima di inviarlo al Client. Questa operazione può avere scopi malevoli, tra cui quello del guadagno economico. In possibile scenario, in questo contesto, è quello che vede l'incremento della grandezza del pacchetto da Agent a Client al fine di maggiorare i consumi della banda stabilita nel contratto iniziale di fornitura del servizio.
	
	\paragraph{Problema 2: scoraggiare abusi della rete}
	Per scoraggiare abusi della rete si utilizza una \textit{prova di lavoro} (Proof of Work) per attestare che il nuovo utente della rete ha speso delle risorse al fine di guadagnare reputazione nel contesto distruibuito.\\
	La reputazione iniziale, all'ingresso della rete distribuita, ha valore zero.
	
	\paragraph{Soluzione problema 1}
	\begin{itemize}
		\item Dati civetta: è richiesta dal Client all'Agent, con frequenza variabile, una risorsa dal valore noto al Client per la verifica di integrità. Se il test risulta negativo si ipotizza la modifica della risorsa da parte dell'Agent.
		\item Dati duplicati: il Client si connette a due Agent diversi e chiede ad entrambi la risorsa desiderata, alla ricezione si comparano per la verifica di integrità. Se il test risulta negativo si ipotizza la modifica della risorsa da parte di uno dei due Agent.\\
		\textbf{Aspetti negativi:} questa soluzione non è applicabile in caso di connessioni con la risorsa che prevedono \textit{stati}.(??)
		\item Hash Dato: il Client si connette ad un Agent al quale chiede una risorsa e successivamente chiede ad un altro Agent l'hash della risorsa stessa. Il Client effettua quindi un hash (stessa funzione hash) della risorsa ottenuta dall'Agent e la compara con l'hash ottenuto dal secondo Agent. Ulteriore implementazione può essere effettuata richiedendo a più agent l'hash della risorsa.\\
		Se il test risulta negativo si ipotizza la modifica della risorsa da parte del primo Agent.\\
		\textbf{Aspetti negativi:} questa soluzione non è applicabile in caso di connessioni con la risorsa che prevedono \textit{stati}.\\
		\textbf{Caso d'uso:} si reputa questo metodo adatto a verificare un Agent al suo primo ingresso nella rete in quanto non è spesso adatto per l'aspetto negativo sopra descritto.		
	\end{itemize}
	
	\paragraph{Soluzione problema 2}
	L'originalità della soluzione proposta è l'operazione di Proof of Work da far compiere al nuovo utente. Si propone come prova di lavoro che l'utente, prima di entrare attivamente a far parte della rete o durante la sua partecipazione, consumi della banda internet.\\
	La Proof of Work consiste quindi nello sfruttare la sua banda di rete per ottenere una risorsa internet, sulla quale viene verificata l'integrità, come descritto nella soluzione \textit{Dato Hash} del problema 1.\\ 
	Più specificatamente, nel caso d'uso si identificano un Agent A, un Client C ed un nuovo utente X che vuole entrare nella rete e deve dimostrare \textit{la prova di lavoro}.\\
	La situazione iniziale prevede una connessione tra A e C. Per verificare che A fornisca dato $d^A$ originale, C richiede a X di ottenere lo stesso dato $d^X$ e di eseguirne la funzione hash così da ottenere  $H(d^X)=h_1$. $h_1$ sarà restituito a C che ha già effettuato la stessa funzione hash su $d^A$ proveniente da A ed ha ottenuto $H(d^A) = h_2$.\\
	Il test di integrità verifica che $h_1 = h_2$.\\
	Se il test ha successo si ha certezza che:
	\begin{enumerate}
		\item il nuovo utente X ha eseguito correttamente la proof of work 
		\item A ha fornito a C un dato originale, a meno che sia A che X applichino le stesse modifiche al dato prima di effettuarne la funzione hash.\\Questa vincolo è da verificare.
	\end{enumerate}
	
	\paragraph{Considerazioni}
	In un contesto di rete distribuita si possono ipotizzare $n$ attori nel ruolo di X e $n$ test di integrità che C effettua per verificare l'integrità del dato ottenuto da A. \textit{riferimento al consenso BFT}\\
	Il risultato dei test condiziona la reputazione di ogni attore nel ruolo di $X$.\\
	Si può ipotizzare l'applicazione della soluzione $m$ volte, tante quante sono le entità per cui effettuare la proof of work, e considerare vincoli come tempo $t$ minimo di esecuzione della proof of work o la quantità minima di dati $q$ che $X$ deve ottenere su cui applicare la funzione hash.\\
	
	
	Considerando una costo ipotetico $c$ associato al consumo di una quantità $q$ di dati per un'unità di tempo $t$ si può impostare una soglia $S = c*q*t$ superata la quale la proof of work si può considerare eseguita.\\
	L'ultima affermazione garantisce l'asimmetria della proof of work (??). (che è: hash difficile da calcolare perchè spendo risorsa di tempo e di banda e facile da verificare dal client perchè effettuo un semplice confronto di un hash)(vedere https://it.wikipedia.org/wiki/Hashcash per similitudini sulla variabile del tempo)\\
	
	
	La reputazione di ogni entità all'interno della rete, sia Agent che Client, subisce una variazione in quanto ciascun esito delle proof of work viene registrata nella blockchain e questo consente di avere una classifica di affidabilità delle entità quasi in tempo reale.\\
	


		\subsection{Impostazione del lavoro}
	parlare di:
	\begin{itemize}
		\item docker (cosa è, comandi base per fare cosa)
		\item robe di networking per dev mode del progetto:
		\begin{itemize}
			\item openvpn
			\item http://www.naturalborncoder.com/virtualization/2014/10/17/understanding-tun-tap-interfaces/
		
		\end{itemize}
	\end{itemize}
	
	\section{Codice}
	l'appicazione myst crea un layer che instrada la comunicazione sul protocollo scelto (openvpn). L'idea è di modificarla affinchè analizzi i pacchetti prima di instradarli e ne esegua delle operazioni. Ad esempio per le immagini ne aggiunge un payload, etc.\\
	
	
	
	\section{conclusioni}
	
	
\end{document}
